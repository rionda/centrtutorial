\section{Conclusions}

\begin{frame}
  \frametitle{What we presented}
  \begin{itemize}
    \pause
    \item Brief survey of the most common measures of centrality
    \item Axioms for centrality
    \item Focussing on closeness and betweenness centrality:
      \begin{itemize}
        \item exact algorithms on static graphs
        \item exact algorithms on dynamic graphs (streaming, on-line, GPU-based)
        \item approximation algorithms for static graphs
        \item approximation algorithms for dynamic graphs
    \end{itemize}
  \end{itemize}
  In each of the above, there are important open questions and directions for
  future work.
\end{frame}

\begin{frame}
  \frametitle{Big Graphs}
  \begin{itemize}
    \pause
    \item \emph{``Big Data''} is a lot of hype and refers to very different
      things depending on the context.
    \pause
    \item However, the unprecedented \emph{volume}, \emph{velocity}, and
      \emph{variety} pose real algorithmic challenges, especially when dealing
      with expressive and complex representations such as graphs.
    \pause
    \item Challenges are opportunities for researchers!
    \pause
    \item \emph{Big graphs require new algorithms}
  \end{itemize}
\end{frame}

\begin{frame}
  \frametitle{Volume requires new algorithms}
  \pause
  \begin{itemize}
    \item  Classic computational complexity:
    \begin{itemize}
      \item Is there a  polynomial time exact algorithm $\mathbf{\rightarrow}$?
        Go for it!
      \item Your problem is \textbf{NP}-Hard $\mathbf{\rightarrow}$ better think
        about approximation algorithms\ldots
    \end{itemize}
  \item Classic computational complexity: polynomial = feasible
  \pause
  \item But is polynomial time really feasible?
    \begin{itemize}
      \item E.g., Brandes algorithm not feasible for $n = 10^9$
    \end{itemize}
  \pause
  \item On big graphs quadratic time is as bad as \textbf{NP}-Hard
  \begin{itemize}
    \item New, finer-grain, complexity theory needed (?)
  \end{itemize}
  \pause
  \item Need for \emph{massively parallel} algorithms, \emph{out-of-core}
    algorithms, \emph{sublinear} algorithms, \emph{approximated} algorithms,
    \emph{randomized} algorithms, etc.
  \end{itemize}
\end{frame}

\begin{frame}
  \frametitle{Velocity requires new algorithms}
  \pause
  \begin{itemize}
    \item  The velocity with which new data keeps \emph{arriving}\ldots
    \pause
    \item \ldots and the velocity with which the information of interest keeps
      \emph{changing}.
    \vfill
    \item In the case of graphs new edges are formed and old edges might
      disappear at very high speed.
    \pause
    \begin{itemize}
      \item How to maintain the centrality score of all vertices continuously
        updated?
    \end{itemize}
    \vfill
    \pause
    \item Velocity requires \emph{streaming} algorithms that only read each data
    point once (or a few time), specialized small-space data structures
    (\emph{sketches}) that maintain basic statistics and can be updated
      on-the-fly, algorithms which are \emph{robust to changes} in the data, etc.
  \end{itemize}
\end{frame}

\begin{frame}
  \frametitle{Variety requires new algorithms}
  \pause
  \begin{itemize}
    \item Variety refers to the \emph{richness of different information types}
      to be mixed in the analysis.
    \vfill
    \pause
    \item Examples in graphs:
      \begin{itemize}
        \pause
        \item Vertices have attributes;
        \pause
        \item Vertices are spatio-temporally localized and keeps moving;
        \pause
        \item Edges have types (colors);
        \pause
        \item Edges have multiple types (a.k.a.~multigraphs, multiplex networks,
          multidimensional networks, etc.);
        \pause
        \item Each edge has associated a time series representing the amount of
          communication (or activity) along the edge per time unit;
        \pause \item ...
      \end{itemize}
    \vfill
    \pause
    \item Semantic richness in the data implies complexity in the knowledge we can
      extract.
    \pause
    \item Applications involving ``multi-structured'' data require the definition
      of \emph{new, ad-hoc, model and patterns} \ldots
    \pause
    \item \ldots and of course, the \emph{algorithms} to extract them,
    \pause
    \item and these new algorithms need to be able to deal with the volume and the
      velocity!
  \end{itemize}
\end{frame}

\begin{frame}
  \frametitle{Big Graphs}
  \begin{itemize}
    \item The \emph{computational complexity} of most existing graph algorithms
      makes them \emph{impractical} in today's networks, which are:
      \begin{itemize}
        \item massive,
        \item information-rich, and
        \item dynamic.
      \end{itemize}
    \item In order to scale graph analysis to \emph{real-world applications} and
      to keep up with their \emph{highly dynamic nature}, we need to
      \emph{devise new approaches} specifically tailored for \emph{modern
        parallel stream processing engines} that run on clusters of
        shared-nothing commodity hardware.
  \end{itemize}
\end{frame}

\begin{frame}
  \frametitle{\bigskip \bigskip  \Huge Thank you!}
  \centering
  \begin{tabular}{c}
    % after \\: \hline or \cline{col1-col2} \cline{col3-col4} ...
    Francesco Bonchi \\  \emph{http://francescobonchi.com} \\ @FrancescoBonchi\\ $\;$ \\
    Gianmarco De Francisci Morales \\ \emph{http://gdfm.me} \\  @gdfm7\\  $\;$ \\
    Matteo Riondato \\ \emph{http://matteo.rionda.to} \\ @teorionda\\  $\;$ \\
  \end{tabular}
  \begin{block}{\centering Slides available at}
    \centering  http://matteo.rionda.to/centrtutorial/
  \end{block}
\end{frame}
