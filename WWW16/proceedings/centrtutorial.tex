\documentclass[pdfpagelabels=false]{sig-alternate-2013} % option is to shut down hyperref warnings
\setlength{\paperheight}{11in} % To shut down hyperref warnings
\usepackage[T1]{fontenc}
\usepackage{lmodern}
\usepackage{amssymb}
\usepackage{booktabs}
\usepackage{hyperref}
\usepackage[numbers,square,sort&compress]{natbib}
\renewcommand{\refname}{References}
\renewcommand{\bibsection}{\subsection{References}}
\renewcommand{\bibfont}{\raggedright}

\permission{Copyright is held by the author/owner(s).}
\conferenceinfo{WWW'16 Companion,}{April 11--15, 2016, Montr\'eal, Qu\'ebec, Canada.}
\copyrightetc{ACM \the\acmcopyr}
\crdata{978-1-4503-4144-8/16/04. \\
Include the http://DOI string/url }
% be sure to update the red text with your assigned DOI from ACM

\clubpenalty=10000
\widowpenalty = 10000

\begin{document}
\title{Centrality Measures on Big Graphs:\\Exact, Approximated, and Distributed Algorithms}

\numberofauthors{3}
\author{
\alignauthor
Francesco Bonchi\\
       \affaddr{ISI Foundation}\\
       \affaddr{Turin, Italy}\\
       \email{francescobonchi@acm.org}
\end{tabular}\begin{tabular}[t]{p{1.3\auwidth}}\centering
Gianmarco~De~Francisci~Morales\\
       \affaddr{Qatar Computing Research Institute}\\
       \affaddr{Doha, Qatar}\\
       \email{gdfm@acm.org}
\alignauthor
Matteo Riondato\titlenote{Main contact.}\\
       \affaddr{Two Sigma Investments}\\
       \affaddr{New York, NY, USA}\\
       \email{matteo@twosigma.com}
}

\date{12 February 2016}
\maketitle

\begin{abstract}
\end{abstract}

\section{Introduction}
Identifying ``important'' nodes or edges in a graph is a fundamental task
in network analysis, with many applications. Several measures, known as
\emph{centrality indices}, have been proposed over the years, each formalizing the
concept of importance in a different way~\citep{Newman10}. Centrality measures
rely on graph properties to quantify importance. For example, betweenness
centrality considers the fraction of shortest paths going through a
node or edge, while the closeness centrality of a node is the average sum of the inverse
of the distance to every other node. Other centrality measures use eigenvectors,
random walks, degrees, or more complex properties, and can usually be extended to sets of nodes.

With the proliferation of huge networks with millions of nodes and billions of
edges, the importance of having scalable algorithms for computing centrality
indices has become more and more evident, and a number of contributions have been
recently made, ranging from heuristics that perform extremely well in
practice, to approximation algorithms with strong probabilistic guarantees,
to scalable algorithms for the MapReduce platform. Moreover, the dynamic nature
of many networks, i.e., the addition and removal of nodes and edges over
time, dictates the need to keep the computed values of centrality up-to-date as
the graph changes. These challenging problems have enjoyed enormous interest
from the research community, resulting in several relevant approaches proposed recently
to tackle them.

Our tutorial presents, in a unified framework, some of the different measures of
centrality, and discusses the algorithms to compute them, both in an exact and in
an approximate way, both in-memory and in a distributed fashion for the
MapReduce framework of computation. Our tutorial represents an effort to ease the
comparison between different measures, the different quality guarantees offered
by approximation algorithms, and the different trade-offs and scalability
behaviors characterizing distributed algorithms. We believe this unity
of presentation is beneficial both for newcomers and for experienced researchers
in the field.

\section{Outline}
The tutorial is structured in three main technical parts, plus a concluding part
where we discuss future research directions. All the three technical parts will
contains both theory and experimental results.

\begin{enumerate}
	\item {\bf Introduction: definitions and exact algorithms}
		\begin{enumerate}
			\item The axioms of centrality~\citep{BoldiV14}
			\item Definitions of centrality~\citep{Newman10}, including, but not
				limited to: betweenness, closeness, degree, eigenvector,
				harmonic, Katz, absorbing random-walk~\citep{MavroforakisMG15},
				and spanning-edge centrality~\citep{MavroforakisGLKT15}.
			\item Betweenness centrality: exact algorithm~\citep{Brandes01} and
				heuristically-faster exact algorithms for betweenness
				centrality~\citep{ErdosIBT15,SariyuceSKC13}.
			\item Exact algorithms for betweenness centrality in a dynamic

				graph~\citep{LeeLPCC12,NasrePR14,PontecorviR15}.
			\item Exact algorithms for closeness centrality in a dynamic
				graph~\citep{SariyuceKSC13b}.
		\end{enumerate}
	\item {\bf Approximation algorithms}
		\begin{enumerate}
			\item Sampling-based algorithm for closeness
				centrality~\citep{EppsteinW04}.
			\item Betweenness centrality: almost-linear-time approximation
				algorithm~\citep{Yoshida14}, basic sampling-based
				algorithm~\citep{BrandesP07}, refined
				estimators~\citep{GeisbergerSS08}, VC-dimension bounds for
				betweenness centrality~\citep{RiondatoK15}.
			\item Approximation algorithms for betweenness centrality in dynamic
				graphs~\citep{KasWCC13,BergaminiMS14,BergaminiM15,HayashiAY15}.
		\end{enumerate}
	\item {\bf Highly-scalable algorithms}
		\begin{enumerate}
			\item GPU-based algorithms~\citep{SariyuceKSC13}.
			\item Exact parallel streaming algorithm for betweenness centrality in a
				dynamic graph~\citep{KourtellisMB15}.
		\end{enumerate}
	\item {\bf Challenges and directions for future research}
\end{enumerate}

\section{Intended Audience}
The tutorial is aimed at researchers interested in the theory and the
applications of algorithms for graph mining and social network analysis.

We do not require any specific existing knowledge. The tutorial is designed for
an audience of computer scientists who have a general idea of the problems and
challenges in graph analysis. We will present the material in such a way that
any advanced undergraduate student would be able to productively follow our
tutorial. %
%and we will actively engage with the audience and adapt our pace and style to ensure
%that every attendee can benefit from our tutorial.
The tutorial starts from the basic definitions and progressively moves to more advanced
algorithms, including sampling-based approximation algorithms and MapReduce
algorithms, so that it will be of interest both to researchers new
to the field and to a more experienced audience.

\section{Duration: \textrm{Half-day}}

\section{Previous editions of the tutorial}
The tutorial was not previously offered. We did not find any tutorial covering
similar topics in the programs of recent relevant conferences.

\section{Support materials}
We are developing a mini-website at
\url{http://matteo.rionda.to/centrtutorial/}. It will contain the
abstract of the tutorial, a detailed outline with short a description of
each item of the outline, a full list of references with links to
electronic editions, a list of software packages implementing the
algorithms, and the slides used in the tutorial presentation. A preliminary version of the
website will be available 15 days after the tutorial is accepted. A preliminary
version of the slides will be available 30 days before the conference, or in any
case by any deadline given to us by the conference organizers, and the final
version will be available 15 days before the conference.

\bibliographystyle{abbrvnat}
\bibliography{centrality}

\end{document}
