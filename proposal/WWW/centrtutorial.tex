\documentclass[11pt]{article}
\usepackage[top=1in, bottom=1.25in, left=1.25in, right=1.25in]{geometry}
\usepackage{lmodern}
\usepackage[T1]{fontenc}
\usepackage{amssymb}
\usepackage{booktabs}
\usepackage{hyperref}
\usepackage[numbers,square,sort&compress]{natbib}
\renewcommand{\refname}{References}
\renewcommand{\bibsection}{\subsection*{References}}
\renewcommand*{\bibfont}{\raggedright}
\usepackage{mdwlist}

\renewcommand{\labelenumi}{\arabic{enumi}.}
\renewcommand{\labelenumii}{\arabic{enumi}.\arabic{enumii}}

\newcommand{\spara}[1]{\smallskip\noindent{\bf #1}}
\newcommand{\para}[1]{\noindent{\bf #1}}

\title{Centrality Measures on Big Graphs:\\Exact, Approximated, and Distributed
Algorithms \\ {\small Proposal for a half-day tutorial at WWW'16}}

\author{
	Francesco Bonchi\footnote{ISI Foundation, Turin Italy.
	\texttt{francescobonchi@acm.org}} \and Gianmarco De
	Francisci Morales\footnote{Aalto University, Helsinki, Finland.
	\texttt{gdfm@acm.org}} \and Matteo
	Riondato\footnote{Two Sigma Investments, New York, NY, USA.
	\texttt{matteo@twosigma.com}. Main contact}
	%\email{francescobonchi@acm.org} \email{gdfm@acm.org} \email{matteo@twosigma.com}
}
\date{}
\begin{document}
\maketitle

\section*{Topic and description}
Identifying ``important'' nodes or edges in a graph is a fundamental task
in network analysis, with many applications. Several measures, known as
\emph{centrality indices}, have been proposed over the years, each formalizing the
concept of importance in a different way~\citep{Newman10}. Centrality measures
rely on graph properties to quantify importance. For example, betweenness
centrality considers the fraction of shortest paths going through a
node or edge, while the closeness centrality of a node is the average sum of the inverse
of the distance to every other node. Other centrality measures use eigenvectors,
random walks, degrees, or more complex properties, and can usually be extended to sets of nodes.

With the proliferation of huge networks with millions of nodes and billions of
edges, the importance of having scalable algorithms for computing centrality
indices has become more and more evident, and a number of contributions have been
recently made, ranging from heuristics that perform extremely well in
practice, to approximation algorithms with strong probabilistic guarantees,
to scalable algorithms for the MapReduce platform. Moreover, the dynamic nature
of many networks, i.e., the addition and removal of nodes and edges over
time, dictates the need to keep the computed values of centrality up-to-date as
the graph changes. These challenging problems have enjoyed enormous interest
from the research community, resulting in several relevant approaches proposed recently
to tackle them.

Our tutorial presents, in a unified framework, some of the different measures of
centrality, and discusses the algorithms to compute them, both in an exact and in
an approximate way, both in-memory and in a distributed fashion for the
MapReduce framework of computation. Our tutorial represents an effort to ease the
comparison between different measures, the different quality guarantees offered
by approximation algorithms, and the different trade-offs and scalability
behaviors characterizing distributed algorithms. We believe this unity
of presentation is beneficial both for newcomers and for experienced researchers
in the field.

\clearpage

\subsection*{Outline}
The tutorial is structured in three main technical parts, plus a concluding part
where we discuss future research directions. All the three technical parts will
contains both theory and experimental results.

\begin{enumerate}
	\item {\bf Introduction: definitions and exact algorithms}
		\begin{enumerate}
			\item The axioms of centrality~\citep{BoldiV14}
			\item Definitions of centrality~\citep{Newman10}, including, but not
				limited to: betweenness, closeness, degree, eigenvector,
				harmonic, Katz, absorbing random-walk~\citep{MavroforakisMG15},
				and spanning-edge centrality~\citep{MavroforakisGLKT15}.
			\item Betweenness centrality: exact algorithm~\citep{Brandes01} and
				heuristically-faster exact algorithms for betweenness
				centrality~\citep{ErdosIBT15,SaryuceSKC13}.
			\item Exact algorithms for betweenness centrality in a dynamic
				graph~\citep{LeeLPCC12,NasrePR14,PontecorviR15}.
			\item Exact algorithms for closeness centrality in a dynamic
				graph~\citep{SariyuceKSC13b}.
		\end{enumerate}
	\item {\bf Approximation algorithms}
		\begin{enumerate}
			\item Sampling-based algorithm for closeness
				centrality~\citep{EppsteinW04}.
			\item Betweenness centrality: almost-linear-time approximation
				algorithm~\citep{Yoshida14}, basic sampling-based
				algorithm~\citep{BrandesP07}, refined
				estimators~\citep{GeisbergerSS08}, VC-dimension bounds for
				betweenness centrality~\citep{RiondatoK15}.
			\item Approximation algorithms for betweenness centrality in dynamic
				graphs~\citep{KasWCC13,BergaminiMS14,BergaminiM15}.
		\end{enumerate}
	\item {\bf Highly-scalable algorithms}
		\begin{enumerate}
			\item GPU-based algorithms~\citep{SariyuceKSC13}.
			\item Exact parallel streaming algorithm for betweenness centrality in a
				dynamic graph~\citep{KourtellisMB15}.
		\end{enumerate}
	\item {\bf Challenges and directions for future research}
\end{enumerate}

\section*{Intended Audience}
The tutorial is aimed at researchers interested in the theory and the
applications of algorithms for graph mining and social network analysis.

We do not require any specific existing knowledge. The tutorial is designed for
an audience of computer scientists who have a general idea of the problems and
challenges in graph analysis. We will present the material in such a way that
any advanced undergraduate student would be able to productively follow our
tutorial. %
%and we will actively engage with the audience and adapt our pace and style to ensure
%that every attendee can benefit from our tutorial.
The tutorial starts from the basic definitions and progressively moves to more advanced
algorithms, including sampling-based approximation algorithms and MapReduce
algorithms, so that it will be of interest both to researchers new
to the field and to a more experienced audience.

\section*{Duration: \textrm{Half-day}}

\section*{Previous editions of the tutorial}
The tutorial was not previously offered. We did not find any tutorial covering
similar topics in the programs of recent relevant conferences.

\vspace{-10pt}
\section*{Organizers}
This tutorial is developed by Francesco Bonchi, Gianmarco De Francisci Morales,
and Matteo Riondato. All three instructors will attend the conference.

\para{Francesco Bonchi} is Research Leader at the ISI Foundation, Turin, Italy,
where he leads the "Algorithmic Data Analytics" group. He is also Scientific
Director for Data Mining at Eurecat (Technological Center of Catalunya),
Barcelona. Before he was Director of Research at Yahoo Labs in Barcelona, Spain,
leading the Web Mining Research group.

His recent research interests include mining query-logs, social networks, and
social media, as well as the privacy issues related to mining these kinds of
sensible data.
%In the past he has been interested in data mining query
%languages, constrained pattern mining, mining spatiotemporal and mobility data,
%and privacy preserving data mining.

He will be PC Chair of the 16th IEEE International Conference on Data Mining
(ICDM 2016) to be held in Barcelona in December 2016. He is member of the ECML
PKDD Steering Committee, Associate Editor of the newly created IEEE Transactions
on Big Data (TBD), of the IEEE Transactions on Knowledge and Data Engineering
(TKDE), the ACM Transactions on Intelligent Systems and Technology (TIST),
Knowledge and Information Systems (KAIS), and member of the Editorial Board of
Data Mining and Knowledge Discovery (DMKD). %
%He has been program co-chair of the
%European Conference on Machine Learning and Principles and Practice of Knowledge
%Discovery in Databases (ECML PKDD 2010). Dr. Bonchi has also served as program
%co-chair of the first and second ACM SIGKDD International Workshop on Privacy,
%Security, and Trust in KDD (PinKDD 2007 and 2008), the 1st IEEE International
%Workshop on Privacy Aspects of Data Mining (PADM 2006), and the 4th
%International Workshop on Knowledge Discovery in Inductive Databases (KDID
%2005).
%He is co-editor of the book "Privacy-Aware Knowledge Discovery: Novel
%Applications and New Techniques" published by Chapman \& Hall/CRC Press.
%He earned his Ph.D. in computer science from the University of Pisa in December 2003.
He presented a tutorial at ACM KDD'14.

\para{Gianmarco De Francisci Morales} is a Visiting Scientist at Aalto
University. Previously he worked as a Research Scientist at Yahoo Labs
Barcelona, and as a Research Associate at ISTI-CNR in Pisa. His research focuses
on scalable data mining, with an emphasis on Web mining and data-intensive
scalable computing systems.
%He is an active member of the open source community of the Apache Software Foundation, working on the Hadoop ecosystem, and a committer for the Apache Pig project.
He is one of the lead developers of Apache SAMOA, an open-source platform for
mining big data streams. He presented a tutorial on stream mining at IEEE
BigData'14.

\para{Matteo Riondato} is a Research Scientist in the Labs group at Two Sigma
Investments. Previously he was a postdoc at Stanford and at Brown. His
dissertation on sampling-based randomized algorithms for data and graph mining
received the Best Student Poster Award at SIAM SDM'14. His research focuses on
exploiting advanced theory to develop practical algorithms for time series
analysis, pattern mining, and social network analysis. He presented
tutorials at ACM KDD'15, ECML PKDD'15, and ACM CIKM'15.

\subsection*{Support materials}
We are developing a mini-website at
\url{http://matteo.rionda.to/centrtutorial/}. It will contain the
abstract of the tutorial, a detailed outline with short a description of
each item of the outline, a full list of references with links to
electronic editions, a list of software packages implementing the
algorithms, and the slides used in the tutorial presentation. A preliminary version of the
website will be available 15 days after the tutorial is accepted. A preliminary
version of the slides will be available 30 days before the conference, or in any
case by any deadline given to us by the conference organizers, and the final
version will be available 15 days before the conference.

%\bibliographystyle{abbrvnat}
%\bibliography{centrtutorial}

\setlength{\bibsep}{0pt}
\begin{thebibliography}{22}
\providecommand{\natexlab}[1]{#1}
\providecommand{\url}[1]{\texttt{#1}}
\expandafter\ifx\csname urlstyle\endcsname\relax
  \providecommand{\doi}[1]{doi: #1}\else
  \providecommand{\doi}{doi: \begingroup \urlstyle{rm}\Url}\fi

\bibitem[Bergamini and Meyerhenke(2015)]{BergaminiM15}
E.~Bergamini and H.~Meyerhenke.
\newblock Fully-dynamic approximation of betweenness centrality.
\newblock ESA'15, 2015.

\bibitem[Bergamini et~al.(2014)Bergamini, Meyerhenke, and
  Staudt]{BergaminiMS14}
E.~Bergamini, H.~Meyerhenke, and C.~L. Staudt.
\newblock Approximating betweenness centrality in large evolving networks.
\newblock ALENEX'15, 2015.

\bibitem[Boldi and Vigna(2014)]{BoldiV14}
P.~Boldi and S.~Vigna.
\newblock Axioms for centrality.
\newblock \emph{Internet Mathematics}, 10\penalty0 (3--4):\penalty0 222--262,
  2014.

\bibitem[Brandes(2001)]{Brandes01}
U.~Brandes.
\newblock A faster algorithm for betweenness centrality.
\newblock \emph{J. Math. Sociol.}, 25\penalty0 (2):\penalty0 163--177, 2001.

\bibitem[Brandes and Pich(2007)]{BrandesP07}
U.~Brandes and C.~Pich.
\newblock Centrality estimation in large networks.
\newblock \emph{Int. J. Bifurcation and Chaos}, 17\penalty0 (7):\penalty0
  2303--2318, 2007.

\bibitem[Eppstein and Wang(2004)]{EppsteinW04}
D.~Eppstein and J.~Wang.
\newblock Fast approximation of centrality.
\newblock \emph{J. Graph Algorithms Appl.}, 8\penalty0 (1):\penalty0 39--45,
  2004.

\bibitem[Erd\H{o}s et~al.(2015)Erd\H{o}s, Ishakian, Bestavros, and
  Terzi]{ErdosIBT15}
D.~Erd\H{o}s, V.~Ishakian, A.~Bestavros, and E.~Terzi.
\newblock A divide-and-conquer algorithm for betweenness centrality.
\newblock SDM'15, 2015.

\bibitem[Geisberger et~al.(2008)Geisberger, Sanders, and
  Schultes]{GeisbergerSS08}
R.~Geisberger, P.~Sanders, and D.~Schultes.
\newblock Better approximation of betweenness centrality.
\newblock ALENEX'08, 2008

\bibitem[Kas et~al.(2013)Kas, Wachs, Carley, and Carley]{KasWCC13}
M.~Kas, M.~Wachs, K.~M. Carley, and L.~R. Carley.
\newblock Incremental algorithm for updating betweenness centrality in
  dynamically growing networks.
\newblock ASONAM'13, 2013.

\bibitem[Kourtellis et~al.(2015)Kourtellis, Morales, and
  Bonchi]{KourtellisMB15}
N.~Kourtellis, G.~D.~F. Morales, and F.~Bonchi.
\newblock Scalable online betweenness centrality in evolving graphs.
\newblock \emph{{IEEE} Trans. Knowl. Data Eng.}, 27\penalty0 (9):\penalty0
  2494--2506, 2015.

\bibitem[Lee et~al.(2012)Lee, Lee, Park, Choi, and Chung]{LeeLPCC12}
M.-J. Lee, J.~Lee, J.~Y. Park, R.~H. Choi, and C.-W. Chung.
\newblock {QUBE}: A quick algorithm for updating betweenness centrality.
\newblock WWW'12, 2012.

\bibitem[Mavroforakis et~al.(2015{\natexlab{a}})Mavroforakis, Garcia-Lebron,
  Koutis, and Terzi]{MavroforakisGLKT15}
C.~Mavroforakis, R.~Garcia-Lebron, I.~Koutis, and E.~Terzi.
\newblock Spanning edge centrality: Large-scale computation and applications.
\newblock WWW'15, 2015.

\bibitem[Mavroforakis et~al.(2015{\natexlab{b}})Mavroforakis, Mathioudakis, and
  Gionis]{MavroforakisMG15}
C.~Mavroforakis, M.~Mathioudakis, and A.~Gionis.
\newblock Absorbing random-walk centrality: Theory and algorithms.
\newblock CoRR/abs 1509.02533,  2015.

\bibitem[Nasre et~al.(2014)Nasre, Pontecorvi, and Ramachandran]{NasrePR14}
M.~Nasre, M.~Pontecorvi, and V.~Ramachandran.
\newblock Betweenness centrality--incremental and faster.
\newblock MFCS'14, 2014.

\bibitem[Newman(2010)]{Newman10}
M.~E.~J. Newman.
\newblock \emph{Networks -- An Introduction}.
\newblock Oxford University Press, 2010.

\bibitem[Pontecorvi and Ramachandran(2015)]{PontecorviR15}
M.~Pontecorvi and V.~Ramachandran.
\newblock A faster algorithm for fully dynamic betweenness centrality.
\newblock CoRR/abs 1506.05783, 2015.

\bibitem[Riondato and Kornaropoulos(2015)]{RiondatoK15}
M.~Riondato and E.~M. Kornaropoulos.
\newblock Fast approximation of betweenness centrality through sampling.
\newblock \emph{Data Mining Knowl.~Disc.}, (to appear), 2015.

\bibitem[Sariy{\"u}ce et~al.(2013)Sariy{\"u}ce, Kaya, Saule, and
  {\c{C}}ataly{\"u}rek]{SariyuceKSC13}
A.~E. Sariy{\"u}ce, K.~Kaya, E.~Saule, and {\"U}.~V. {\c{C}}ataly{\"u}rek.
\newblock Betweenness centrality on gpus and heterogeneous architectures.
\newblock In \emph{Proceedings of the 6th Workshop on General Purpose Processor
  Using Graphics Processing Units}, pages 76--85. ACM, 2013.

\bibitem[Sariyuce et~al.(2013)Sariyuce, Kaya, Saule, and
  Catalyurek]{SariyuceKSC13b}
A.~E. Sariyuce, K.~Kaya, E.~Saule, and U.~V. Catalyurek.
\newblock Incremental algorithms for closeness centrality.
\newblock BigData'13, 2013.

\bibitem[Sar{\i}y\"{u}ce et~al.(2013)Sar{\i}y\"{u}ce, Saule, Kaya, and
  \c{C}ataly\"{u}rek]{SaryuceSKC13}
A.~E. Sar{\i}y\"{u}ce, E.~Saule, K.~Kaya, and U.~V. \c{C}ataly\"{u}rek.
\newblock Shattering and compressing networks for betweenness centrality.
\newblock SDM'13, 2013.

\bibitem[Yoshida(2014)]{Yoshida14}
Y.~Yoshida.
\newblock Almost linear-time algorithms for adaptive betweenness centrality
  using hypergraph sketches.
\newblock KDD'14, 2014.

\end{thebibliography}
\end{document}
